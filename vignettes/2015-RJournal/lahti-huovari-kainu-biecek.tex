\title{\CRANpkg{eurostat}: Eurostat Open Data R Tools\\DRAFT VERSION IN PROGRESS}
\author{Leo Lahti, Janne Huovari, Markus Kainu, Przemyslaw Biecek}

\maketitle

%An abstract of less than 150 words.
\abstract{Whereas open data released by governmental institutions are
opening up novel opportunities for research and citizen science,
efficient tools to access and analyze these data sets are needed to
realize the full potential of these information resources. We
introduce here the \CRANpkg{eurostat} R package that provides a suite
of tools to access open data from Eurostat, including functions to
search, download, and manipulate Eurostat data in an automated and
reproducible manner. The online documentation provides detailed
examples on how to access, summarize and visualize these
spatio-temporal data sets. The package expands previous related work
and has been extensively tested by the user community. This
contributes to the growing ecosystem of R packages that provide
algorithmic tools for reproducible computational research in social
science and humanities.}

Eurostat\footnote{\url{http://ec.europa.eu/eurostat/data/database}},
the statistical office of the European Union, is providing a rich
collection of European level demographic and economic data through its
open data service, which currently includes over 8800 data sets on
European demography, economics, health, infrastructure, traffic and
other topics. In many cases the statistics are available with great
geographical resolution and as time series spanning over several years
or decades.

The availability of tools to access and analyse data collections from
the public domain can greatly benefit reproducible
research \citep{Gandrud13, Boettiger2015}. When the data resources and
analysis algorithms are openly available, the complete analytical
workflow spanning from raw data to the final publication can be made
fully automated and transparent. Standardization of common data
analysis tasks via dedicated software packages can help to automate
the analysis workflow, greatly facilitating reproducibility and code
sharing, and making the data analysis more efficient. At the same
time, the algorithms need to be customized for variations in data
formats, access details, and typical use cases.

Here, we introduce the \CRANpkg{eurostat} R package that implements R
tools specifically designed to facilitate such automated access to
open data from
Eurostat \footnote{\url{http://ec.europa.eu/eurostat}}. Despite
earlier efforts, a dedicated R package for eurostat open data has been
missing. The eurostat R package introduced here brings together
earlier efforts from our earlier CRAN
packages \CRANpkg{statfi} \citep{statfi} and \CRANpkg{smarterpoland}
\citep{smarterpoland}. Compared to this earlier work, we have combined the
relevant parts of these two packages and implemented an expanded set
of tools with a specific focus on the Eurostat data collection. The
first version of the new \CRANpkg{eurostat} package was released in
CRAN in 2014. It has been actively developed by several contributors
and based on community feedback in Github. We are now reporting the
first mature version of the package that has been improved and tested
by multiple users. The package and its predecessors have been applied
in several case studies by us and others\footnote{See
e.g. http://blog.revolutionanalytics.com/2015/04/financial-times-tracks-unemployment-with-r.html}.

Related work includes the \CRANpkg{datamart} \citep{datamart} and the
\CRANpkg{quandl}
\citep{quandl} R packages that provide generic tools that can be used to access
certain versions of Eurostat data. In contrast to these generic
database packages, our eurostat package provides functionality that is
particularly tailored for the Eurostat open data service. The
development version of another related R package
\pkg{reurostat}\footnote{https://github.com/Tungurahua/reurostat} does not seem to be actively maintained at the moment.
Moreover, our \CRANpkg{eurostat} package depends, imports or suggests
the following external R packages:
\CRANpkg{devtools} \citep{devtools}, \CRANpkg{dplyr} \citep{dplyr},
\CRANpkg{knitr} \citep{knitr}, \CRANpkg{ggplot2} \citep{ggplot2},
\CRANpkg{mapproj} \citep{mapproj}, \CRANpkg{plotrix} \citep{plotrix},
\CRANpkg{reshape2} \citep{reshape2}, \CRANpkg{rmarkdown}
\citep{rmarkdown}, \CRANpkg{stringi} \citep{stringi},
\CRANpkg{testthat} \citep{testthat}, and \CRANpkg{tidyr}
\citep{tidyr}. The \CRANpkg{eurostat} R package is
part of rOpenGov collection
\citep{Lahti13icml} that provides reproducible research tools for
computational social science and digital humanities.

In summary, the \CRANpkg{eurostat} package provides custom tools to
search, retrieve, modify and visualize data from the Eurostat open
data service. The package supports key features such as data cache,
date formatting, and tidy data principles \citep{wickham2014} using
the the \CRANpkg{tidyr} R package \citep{tidyr}. Here, we provide an
overview of the core functionality in the current CRAN release version
(1.2.1). For further examples, see the package
vignette\footnote{https://github.com/rOpenGov/eurostat/vignette/eurostat\_tutorial.Rmd}.


\section{Search and download commands}

To install and load the CRAN release version, just type in R:

\begin{example}
> install.packages("eurostat")
> library("eurostat")
\end{example}

The complete table of contents of the database can be browsed
on-line\footnote{http://ec.europa.eu/eurostat/data/database}, or
downloaded in R with the command \code{toc <- get\_eurostat\_toc()}. The function \code{search\_eurostat()} is used
to make a more focused search over the table of contents. To retrieve
data for 'Modal split of passenger transport', for instance, use:

\begin{example}
> query <- search_eurostat("Modal split of passenger transport", type = "table")
\end{example}

The \code{type} argument limits the search on a selected data set type
in the above example. The options for this argument
include \dfn{'table'}, \dfn{'dataset'} or \dfn{'folder'}, referring to
different levels of hierarchy in the data organization: a \code{table}
resides in \code{dataset}, which is in turn stored in a
\code{folder}.

Values in the \code{code} column of the \code{search\_eurostat()}
function output provide data sets identifiers that can be used in
subsequent download commands. Alternatively, these identifier codes
can be browsed at the Eurostat open data service; check the codes in
the Data Navigation Tree listed after each dataset in parentheses. Let
us look at the data set identifier and title for the first entry of
the query data:

\begin{example}
> query$code[[1]]
[1] "tsdtr210"

> query$title[[1]]
[1] "Modal split of passenger transport"
\end{example}


To retrieve the data set with this identifier, use

\begin{example}
> dat <- get_eurostat(id = "tsdtr210", time_format = "num")
\end{example}

As the original data is annual in this example, we have selected a
numeric time format. This is more convient for annual time series than
the default date format. The data sets are provided as standard data
frames to support standard tools for data subsetting and
reshaping. The above function call returns a table on transport
statistics.  The first lines of the output are shown in
Table~\ref{tab:getdatatable}.

\begin{table}[ht!]
\centering
\begin{tabular}{rlllrr}
\toprule
  \hline
 & unit & vehicle & geo & time & values \\ 
  \hline
  1 & PC & BUS\_TOT & AT & 1990.00 & 11.00 \\ 
  2 & PC & BUS\_TOT & BE & 1990.00 & 10.60 \\ 
  3 & PC & BUS\_TOT & BG & 1990.00 &  \\ 
  4 & PC & BUS\_TOT & CH & 1990.00 & 3.70 \\ 
  5 & PC & BUS\_TOT & CY & 1990.00 &  \\ 
  6 & PC & BUS\_TOT & CZ & 1990.00 &  \\ 
   \hline
\bottomrule   
\end{tabular}
\caption{First lines of output from the \code{get\_eurostat()} function for the data set with the identifier 'tsdtr210'.}
\label{tab:getdatatable}
\end{table}

\begin{table}[h!]
\centering
\begin{tabular}{rlllrr}
\toprule
  \hline
  & unit & vehicle & geo & time & values \\ 
  \hline
  1 & Percentage & Motor coaches, buses and trolley buses & Austria & 1990.00 & 11.00 \\ 
  2 & Percentage & Motor coaches, buses and trolley buses & Belgium & 1990.00 & 10.60 \\ 
  3 & Percentage & Motor coaches, buses and trolley buses & Bulgaria & 1990.00 &  \\ 
  4 & Percentage & Motor coaches, buses and trolley buses & Switzerland & 1990.00 & 3.70 \\ 
  5 & Percentage & Motor coaches, buses and trolley buses & Cyprus & 1990.00 &  \\ 
  6 & Percentage & Motor coaches, buses and trolley buses & Czech Republic & 1990.00 &  \\ 
   \hline 
\bottomrule     
\end{tabular}
\caption{The output from \code{get\_eurostat()} (see Table~\ref{tab:getdatatable}), converted into human-readable labels retrieved by \code{label\_eurostat()}.}
\label{tab:getdatatable2}
\end{table}

The entries in many columns of the Table~\ref{tab:getdatatable} are as
such not readily interpretable. To improve interpretability, these
variable identifiers can be replaced with human-readable labels with
the \code{label\_eurostat()} function; the simple
call \code{label\_eurostat(dat)} converts the original identifiers
into a human-readable version (Table~\ref{tab:getdatatable2}) based on
translations available from the Eurostat database.

The downloaded data sets are stored in cache by default to avoid
repeated downloads of identical data sets. This can speed up the
analysis. Moreover, storing an exact copy of the retrieved raw data on
the hard disk supports reproducibility when the source database is
constantly updated.


\section{Visualizing the Eurostat data}

The transport data set in our example includes three classes of
vehicles. Three-dimensional data sets such as this can be conveniently
visualized as triangular maps by using
the \CRANpkg{plotrix} \citep{plotrix} package. The
Figure~\ref{fig:transport}A illustrates the distribution of vehicle
types in different countries. Interestingly, the Eurostat data also
reveals a decreasing trend of road accidents in many countries over
time (Figure~\ref{fig:transport})B as described in more detail in our
recent blog
post\footnote{http://pbiecek.github.io/archivist/justGetIT.html}.

\begin{figure}
\setkeys{Gin}{width=0.5\textwidth}
\begin{center}
\begin{tabular}{cc}
\includegraphics{2015-manu-search2-1}
\includegraphics{2015-manu-roadacc-1}
\end{tabular}
\end{center}
\caption{{\bf A} Passenger transport data retrieved with the \CRANpkg{eurostat} package visualized on a triangular \CRANpkg{plotrix} map. {\bf B} Time series of the number of people killed in road accidents.}
\label{fig:transport}
\end{figure}



The Eurostat database includes a variety of demographic and health
indicators. We see, for instance, that overweight varies remarkably
across different age groups quantified by the body-mass index (BMI)
(Figure~\ref{fig:bmi})\footnote{The fully reproducible source code
of this example is available at
https://github.com/rOpenGov/eurostat/blob/master/vignettes/2015-RJournal/lahti-huovari-kainu-biecek.md}.


\begin{figure}
\setkeys{Gin}{width=0.5\textwidth}
\begin{center}
\includegraphics{2015-manu-bmi-1}
\end{center}
\caption{The body-mass index in different age groups based on Eurostat open data.}
\label{fig:bmi}
\end{figure}



\section{Geospatial information}

The indicators in the Eurostat open data service are typically
available as annual time series grouped by country, and in some cases
at more refined temporal or geographic levels. Importantly, Eurostat
provides complementary geospatial data on administrative statistical
units, available as standard
shapefiles\footnote{http://ec.europa.eu/eurostat/web/gisco/geodata/reference-data/administrative-units-statistical-units}. These
can be used to visualize the available statistics on the European map
at the appropriate geographic resolution.

As an example of geospatial data visualization, let us look at
disposable income of private households (data set identifier
tgs00026\footnote{http://ec.europa.eu/eurostat/en/web/products-datasets/-/TGS00026}). This
data set is available at the geographic level of NUTS2 regions, which
is the intermediate level of territorial units in the Eurostat
regional classifications, roughly correspond to provinces or states in
each country\footnote{http://ec.europa.eu/eurostat/web/nuts/overview}
(Figure~\ref{fig:mapexample}).

A detailed treatment of this example, together with reproducible
source code, is available
on-line\footnote{http://ropengov.github.io/r/2015/05/01/eurostat-package-examples}. The
example demonstrates how the Eurostat data sets and geospatial data
(shapefiles), retrieved with the eurostat package, can be combined
with additional map visualization tools and other utilities including
\CRANpkg{grid} \citep{grid}, \CRANpkg{maptools} \citep{maptools}, \CRANpkg{rgdal} \citep{rgdal},
\CRANpkg{rgeos} \citep{rgeos}, \CRANpkg{scales} \citep{scales}, and
\CRANpkg{stringr} \citep{stringr}.


\begin{figure}
\setkeys{Gin}{width=0.8\textwidth}
\begin{center}
\includegraphics{2015-manu-mapexample-1}
\caption{Disposable income of private households across NUTS2-level national regions in European countries visualized based on geospatial data available from Eurostat.}
\label{fig:mapexample}
\end{center}
\end{figure}


To further facilitate the analysis and visualization of standard
European areas, we have included ready-made country code lists for
certain standard areas. To retrieve, for instance, the eurostat
country codes for EFTA countries (Table~\ref{tab:efta}), we can use:

\begin{example}
data(efta_countries)
\end{example}

Similar country listings are available for EFTA (efta\_countries),
Euro area (ea\_countries), EU (eu\_countries) and EU candidate
countries (candidate\_countries). These auxiliary data sets facilitate
fast selection of specific country groups. The full name and a
two-letter identifier is provided for each country as available from
the Eurostat database. The country codes follow the ISO 3166-1 alpha-2
standard, except that GB and GR are replaced by UK (United Kingdom)
and EL (Greece) in the Eurostat database, respectively. Linking these
country codes with external data sets can be facilitated by
conversions between different country coding standards with
the \CRANpkg{countrycode} R package \citep{countrycode}.


\begin{table}[ht]
\centering
\begin{tabular}{rll}
\toprule
  \hline
  & code & name \\ 
  \hline
  1 & IS & Iceland \\ 
  2 & LI & Liechtenstein \\ 
  3 & NO & Norway \\ 
  4 & CH & Switzerland \\ 
   \hline
\bottomrule   
\end{tabular}
\label{tab:efta}
\caption{The EFTA country code table from the eurostat R package.}
\end{table}



\section{Summary}

The eurostat R package provides convenient tools to access open data
from Eurostat. Combining programmatic access to the data sets with
further analysis and visualization tools allows a seamless and
reproducible automation of the complete data analytical workflow from
accessing the raw data to statistical analysis and final
publication. The full source code of the figures and tables of this
manuscript are available at the package Github
site\footnote{https://github.com/rOpenGov/eurostat/blob/master/vignettes/2015-RJournal/lahti-huovari-kainu-biecek.md}. The
Rmarkdown document provides transparent documentation with full
algorithmic details on the analyses, and can be updated when new
versions of the Eurostat data become available.

The eurostat package provides an example of automated data retrieval
from institutional data repositories, featuring options such as
search, subsetting and cache. Possible future extensions and
improvements include implementation of specific data representation
formats to harmonize the data representation across similar data
sources and to facilitate subsequent tool development. In particular,
we should take further advantage of the existing spatiotemporal data
structures available in R, such as those provided by
the \CRANpkg{spacetime} package \cite{xxx}, and construct wrapper
functions to speed up routine operations such as visualizing the
temporal and geospatial data sets from Eurostat.

To install the latest development version of the eurostat package,
follow the instructions at the github
site\footnote{https://github.com/rOpenGov/eurostat}. The package
source code can be freely used, modified and distributed under the
BSD-2-clause (modified FreeBSD) license. We welcome issues, bug
reports and other feedback via the development
site\footnote{https://github.com/ropengov/eurostat}.


\section*{Acknowledgements}

We are grateful to Eurostat\footnote{http://ec.europa.eu/eurostat} for
maintaining the open data portal and the
rOpenGov\footnote{https://github.ropengov.io} for supporting package
development. This work has been partially funded by Academy of Finland
(decision 293316). We also wish to thank Juuso Parkkinen and Joona
Lehtom{\"a}ki for feedback.


\bibliography{lahti-huovari-kainu-biecek}

\address{Leo Lahti\\
  Department of Mathematics and Statistics\\
  PO Box 20014 University of Turku\\
  Finland\\}
\email{leo.lahti@iki.fi}

\address{Janne Huovari\\
  Affiliation\\
  Address\\
  Country\\}
\email{author2@work}

\address{Markus Kainu\\
  Affiliation\\
  Address\\
  Country\\}
\email{author3@work}

\address{Przemyslaw Biecek\\
  Faculty of Mathematics, Informatics, and Mechanics\\
  University of Warsaw\\
  Banacha 2, 02-097 Warsaw\\
  Poland\\}
\email{P.Biecek@mimuw.edu.pl}

