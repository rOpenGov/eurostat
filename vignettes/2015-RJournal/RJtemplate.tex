% !TeX root = RJwrapper.tex
\title{eurostat: R tools for eurostat open data}
\author{Leo Lahti, Janne Huovari, Markus Kainu, Przemyslaw Biecek}

\maketitle

\abstract{
An abstract of less than 150 words.
}



Introductory section which may include references in parentheses
\citep{R}.

Governmental institutions have started to release increasing amount of
their data resources for the public as open data in the recent
years. This is opening novel opportunities for research and citizen
science. Eurostat, for instance, provides a rich collection of
demographic and economic data through its open data portal. The portal
currently includes ... data sets on ... between years ...

Efficient tools to access and analyse such data collections can
greatly benefit reproducible research. When the data is available, the
analytical methodology spanning from raw data to the final publication
can also be made available following reproducible research principles
\citep{Gandrud13}. Standardization and automatization of
common data analysis tasks via dedicated software packages can greatly
facilitate reproducibility and code sharing, making the research more
transparent and efficient.

Here, we introduce the eurostat R package that implements automated
tools for R to access open data from
Eurostat \footnote{http://ec.europa.eu/eurostat} in
R \citep{R}. The package has been actively developed by several
independent contributors and based on the community feedback in
Github, and its first CRAN release was on.. We are now reporting the
first mature version of the package that has been improved and tested
by multiple users, and includes features like cache, handling dates,
and using the tidy data principles \citep{wickham2014}, with the help of
the tidyr R package \citep{tidyr}.

Despite many efforts to this direction, a dedicated R package for
eurostat open data has been missing. Our work extends our earlier CRAN
packages statfi \citep{statfi} and
smarterpoland \citep{smarterpoland}. Compared to this earlier work, we
have now implemented an expanded set of tools specifically focusing on
the eurostat data collection. The datamart \citep{datamart} and the
quandl \citep{quandl} R packages provide also access to certain
versions of eurostat data. In contrast to these generic database
packages, our work is fully focused on the Eurostat open data portal
and provides specific functionality suited for this data
collection. There is a development version for R package
reurostat\footnote{https://github.com/Tungurahua/reurostat} but it
does not seem to be actively maintained at the moment. The eurostat R
package takes advantage of the following external R packages:
devtools \citep{devtools}, dplyr \citep{dplyr}, knitr \citep{knitr},
ggplot2 \citep{ggplot2}, mapproj \citep{mapproj},
plotrix \citep{plotrix}, reshape2 \citep{reshape2},
rmarkdown \citep{rmarkdown}, stringi \citep{stringi},
testthat \citep{testthat}, tidyr \citep{tidyr}. The eurostat R package
is part of the rOpenGov project
\citep{Lahti13icml}, which provides reproducible research
tools for computational social science and digital humanities.


\section{Overview of the functionality}



The package includes tools to search and retrieve specific data sets
from the Eurostat open data portal, converting identifiers in
human-readable formats, selecting, modifying and visualizing the
data. Further examples are provided in the package
vignette\footnote{https://github.com/rOpenGov/eurostat/vignette/eurostat\_tutorial.Rmd},
and a blog
post\footnote{http://ropengov.github.io/r/2015/05/01/eurostat-package-examples}.

The unified interface to data sets can make data analysis more
straightforward and transparent by providing a standardized and
automated way to access the data sets. Here we describe the
functionality of the current CRAN release version (1.2.1). To install
this, simply use

\begin{Schunk}
\begin{Sinput}
> install.packages("eurostat")
\end{Sinput}
\end{Schunk}

You can download the complete table of contents of the database with the function `get\_eurostat\_toc()`, or use `search\_eurostat()` to make a more focused search over the table of contents. To retrieve data sets for 'disposable income', for instance, use:

\begin{Schunk}
\begin{Sinput}
> library(eurostat)
> income <- search_eurostat("disposable income", type = "dataset")
\end{Sinput}
\end{Schunk}

The data type to search for is also specified. The options include 'table', 'dataset' or 'folder', referring to different levels of hierarchy in data organization: a 'table' resides in 'dataset' that are stored in a 'folder'. You can focus the search on a selected type.

The first lines of the output are shown above. Use values from the `code` column to refer to a specific data set in the subsequent download commands. The dataset identifier codes can also be browsed at the Eurostat
database\footnote{http://ec.europa.eu/eurostat/data/database}, which gives
codes in the Data Navigation Tree after every dataset in parenthesis.

Download the selected data with `get\_eurostat()`. To retrieve the data set with a particular identifier, use 

\begin{Schunk}
\begin{Sinput}
> dat <- get_eurostat(id, time_format = "num")
> print(xtable(head(dat)))
\end{Sinput}
% latex table generated in R 3.2.2 by xtable 1.8-0 package
% Mon Nov 23 13:41:40 2015
\begin{table}[ht]
\centering
\begin{tabular}{rlllrr}
  \hline
 & unit & vehicle & geo & time & values \\ 
  \hline
1 & PC & BUS\_TOT & AT & 1990.00 & 11.00 \\ 
  2 & PC & BUS\_TOT & BE & 1990.00 & 10.60 \\ 
  3 & PC & BUS\_TOT & BG & 1990.00 &  \\ 
  4 & PC & BUS\_TOT & CH & 1990.00 & 3.70 \\ 
  5 & PC & BUS\_TOT & CY & 1990.00 &  \\ 
  6 & PC & BUS\_TOT & CZ & 1990.00 &  \\ 
   \hline
\end{tabular}
\end{table}\end{Schunk}

The original data is annual in this example, hence we have here
selected a numeric time variable as it is more convient for annual
time series than the default date format. To improve the
interpretability of the output, the eurostat variable identifiers
could be further replaced with human-readable labels based on
definitions from Eurostat dictionaries with the `label\_eurostat()`
function. The data is provided in the standard data.frame format, and
all standard tools for data subsetting and reshaping can be
conveniently applied.

The downloaded data sets are stored in cache by default. This can help
to avoid repeated downloading of identical data and helps to speed up
the analysis. Another advantage is that by storing an exact copy of
the data on the hard disk, it is possible to reproduce the analysis
results afterwards even if the source database has been updated.

Data sets containing geographic information can be visualized on a
map. Most indicators are of country-year -type, although some
indicators have data also at lower level of regional breakdown
%\footnote{http://ec.europa.eu/eurostat/ramon/nomenclatures/index.cfm\?TargetUrl=DSP\_GEN\_DESC\_VIEW\_NOHDR\&StrNom=NUTS\_33\&StrLanguageCode=EN}.

\begin{figure}
\setkeys{Gin}{width=0.5\textwidth}
\begin{center}
\includegraphics{RJtemplate-2015-manu-mapexample}
\caption{Test caption.}
\label{fig:mapexample}
\end{center}
\end{figure}


The disposable income of private households at
NUTS2\footnote{http://en.wikipedia.org/wiki/Nomenclature\_of\_Territorial\_Units\_for\_Statistics}
level can be visualized, for instance, as in
Figure~\ref{fig:mapexample}. For a more detailed treatment of this
example, see our related blog
post\footnote{http://ropengov.github.io/r/2015/05/01/eurostat-package-examples}.


Here, we have combined the data retrieved with the eurostat package
with additional map visualization tools and utilities including
grid \citep{grid}, maptools \citep{maptools}, rgdal \citep{rgdal},
rgeos \citep{rgeos}, scales \citep{scales}, and
stringr \citep{stringr}.

Example on spatio-temporal data visualization: let us look at the indicator tgs00026\footnote{http://ec.europa.eu/eurostat/en/web/products-datasets/-/TGS00026}, (Disposable income of private households by NUTS 2 regions) from Eurostat. We are looking at the disposable household income. In addition to downloading and manipulating data from EUROSTAT, we demonstrate how to access and use shapefiles of Europe published by EUROSTAT at Administrative units / Statistical units\footnote{http://ec.europa.eu/eurostat/web/gisco/geodata/reference-data/administrative-units-statistical-units}.



\section{Further examples}

To make a triangle map from the plotrix \citep{plotrix}
package provides an example on visualizing passenger transport data
distributions (Figure~\ref{fig:plotrix}):

\begin{figure}
\setkeys{Gin}{width=0.5\textwidth}
\begin{center}
\includegraphics{RJtemplate-2015-manu-searchdata}
\end{center}
\caption{Test caption2.}
\label{fig:plotrix}
\end{figure}

To facilitate fast plotting of standard European geographic areas, the package provides ready-made lists of the country codes used in the eurostat database for EFTA (efta\_countries), Euro area (ea\_countries), EU (eu\_countries) and EU candidate countries (candidate\_countries). This helps to select specific groups of countries for closer investigation. For conversions with other standard country coding systems, see the countrycode R package \citep{countrycode}. To retrieve the eurostat country code list for EFTA, for instance, use:

\begin{Schunk}
\begin{Sinput}
> data(efta_countries)
> print(xtable(efta_countries))
\end{Sinput}
% latex table generated in R 3.2.2 by xtable 1.8-0 package
% Mon Nov 23 13:41:58 2015
\begin{table}[ht]
\centering
\begin{tabular}{rll}
  \hline
 & code & name \\ 
  \hline
1 & IS & Iceland \\ 
  2 & LI & Liechtenstein \\ 
  3 & NO & Norway \\ 
  4 & CH & Switzerland \\ 
   \hline
\end{tabular}
\end{table}\end{Schunk}


\section{Applications}

The package or its predecessors have already been applied in several case studies by us and independent developers. Financial Times, for instance, have used R to access data from Eurostat\footnote{http://blog.revolutionanalytics.com/2015/04/financial-times-tracks-unemployment-with-r.html} using functions from SmarterPoland, the direct predecessor of our revised and expanded eurostat package.

The archivist R package\footnote{http://pbiecek.github.io/archivist} for archivisation of objects has exemplified\footnote{http://pbiecek.github.io/archivist/justGetIT.html} its functionality by using eurostat to plot the number of people killed by road accidents, showing a decreasing trend of road accidents in many countries (Figure~\ref{fig:roadaccidents}).

We can also look at the distribution of BMI between different age groups (Figure~\ref{fig:bmi}).


\begin{figure}
\setkeys{Gin}{width=0.5\textwidth}
\begin{center}
\includegraphics{RJtemplate-2015-manu-roadaccidents}
\end{center}
\caption{Road accidents caption}
\label{fig:roadaccidents}
\end{figure}


\begin{figure}
\setkeys{Gin}{width=0.5\textwidth}
\begin{center}
\includegraphics{RJtemplate-2015-manu-bmi}
\end{center}
\caption{BMI caption}
\label{fig:bmi}
\end{figure}




\section{Summary}

The eurostat R package provides convenient tools to access open data
from Eurostat. When automated access to the data sets is integrated
with data analytical tools from other packages, this allows a seamless
automation of the data analytical process from raw data access to
statistical analysis and the final publication.

The package source code can be freely used, modified and distributed
under the BSD-2-clause (modified FreeBSD) license. A reproducible
version of this article is available at .. and can be used to generate
the manuscript text along with up-to-date figures and tables with the
latest version of the eurostat data. Manuscript automation provides
transparent documentation with full algorithmic details on how to
access, preprocess, analyse, and report data and analyses, thus
serving a template for good reproducible research practice. The reproducible source code for this manuscript is available at eurostat github page\footnote{https://github.com/rOpenGov/eurostat/blob/master/vignettes/manuscript.Rmd}.

The package exemplifies also the challenges and possible solutions to
reproducible research and automated open data retrieval. Possible
future extensions and improvements include design of specific data
representation class structures. This could facilitate harmonization
of the data representation with similar governmental data sets and
subsequent tool development. The latest development version of the
package can be installed from Github by following the instructions at
the github site\footnote{https://github.com/rOpenGov/eurostat}. We welcome
issues, bug reports and other feedback via the development
site\footnote{https://github.com/ropengov/eurostat}.


\section*{Acknowledgements}

We are grateful to Eurostat\footnote{http://ec.europa.eu/eurostat} for
maintaining the open data portal and the
rOpenGov\footnote{https://github.ropengov.io} for supporting 
package development. This work has been partially funded by Academy of
Finland (decision 293316). We also wish to thank Juuso Parkkinen and Joona Lehtomaki for their feedback on this work.


\bibliography{RJreferences}

\address{Leo Lahti\\
  Department of Mathematics and Statistics\\
  PO Box 20014 University of Turku\\
  Finland\\}
\email{leo.lahti@iki.fi}

\address{Janne Huovari\\
  Affiliation\\
  Address\\
  Country\\}
\email{author2@work}

\address{Markus Kainu\\
  Affiliation\\
  Address\\
  Country\\}
\email{author3@work}

\address{Przemyslaw Biecek\\
  Faculty of Mathematics, Informatics, and Mechanics\\
  University of Warsaw\\
  Banacha 2, 02-097 Warsaw\\
  Poland\\}
\email{P.Biecek@mimuw.edu.pl}
